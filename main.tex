\documentclass[a4paper, 12pt]{article}
\usepackage[utf8]{inputenc}
\usepackage{amsmath,amsthm,amssymb,amsfonts,bbm,bm}
\usepackage[parfill]{parskip}

\title{trend-calculus}
\author{Johannes Graner \and Raazesh Sainudiin \and whomsoeverCoauthor(s)}

\date{August 2020}

\begin{document}

\maketitle

\section{Introduction}

What does trend mean in time series... what are trend reversals and why do we want to detect trend-reversals? % keep it brief and point to wikipedia or other publicly available resources using \footnote{\url{...}}


\section{Data}

What is a typical dataset here and what specific ones are we using -- yfinance, fx1M, etc. Also point o github lamastex forks of these 

\section{Algoritm}

GitHub link to spark-trend-calculus... and Andrew's Lua version...

\section{Mathematical interpretation}

\subsection{Algorithm}

The input to the algorithm is a time series with $N$ observations, 

\[ T = \{ p_i = (t_i, x_i) \in \mathbb{R}^2 : t_i < t_{i+1} \forall i = 1, \dots, N-1 \} \]

and a window size $s \in [2, N] \cap \mathbb{Z}$.

To create the windows in the algorithm, we define the function $f_s : T^s \to T^2$,

\begin{align*}
    f_s(p_{i_1}, \dots, p_{i_s}) = ( & \min_t (\min_x \{ (t_{i_j}, x_{i_j}) : j = 1,\dots,s \}), \\
    & \max_t (\max_x \{ (t_{i_j}, x_{i_j}) : j = 1,\dots,s \}) )
\end{align*}

which finds the earliest minimum and latest maximum among the points $p_{i_1},\dots,p_{i_s}$.

Assume that $N$ is divisible by $s$ (otherwise, only consider $N - \lceil\frac{N}{s} \rceil$ points of the series). Let $n = \frac{N}{s} \in \mathbb{Z}_{\ge 2}$ and let 

\begin{equation*}
    W = \{ w_k = (l_k, h_k) = f_s(p_{1+sk},\dots,p_{s+sk}) : k = 1,\dots,n\}. 
\end{equation*}

Then $W$ is the set of windows that summarize the time series $T$. For each $w_k \in W$, define the trend for that window as 

\begin{equation*}
    \tau_k = \begin{cases}
        1, & k = 1 \\
        \tau(w_{k-1}, w_k), & k > 1
    \end{cases}
\end{equation*}

where $\tau : W^2 \to \{-1,0,1\}$ is given by

\begin{equation*}
    \tau(w_{k-1}, w_k) = sign(sign \circ \pi_x(l_k - l_{k-1}) + sign \circ \pi_x(h_k - h_{k-1}))
\end{equation*}

$\pi_x : \mathbb{R}^2 \to \mathbb{R}$ is the projection $(t,x) \mapsto x$.

To correct the trend when $\tau_k = 0$, let 

\begin{equation*}
    w_k^* = \begin{cases}
        w_{k - \#\{j : j<k, t_j = 0\}}, & \tau_k \ne 0 \\
        (\min_x\{a,b\}, \max_x\{a,b\}), & \tau_k = 0
    \end{cases}
\end{equation*}

where $a = \max_t\{l_{k-1}, h_{k-1}\}, b = \min_t\{l_k, h_k\}$. This creates intermediate windows at every point where the trend is 0 and inserts them at the correct places in the sequence $(w_k^*)$. The length of this sequence is $n^* = n + \#\{k : \tau_k = 0\}$. This sequence is then used to define the corrected trends $\tau_k^*$.

\begin{equation*}
    \tau_k^* = \begin{cases}
        1, & k = 1 \\
        \tau(w_{k-1}^*, w_k^*), & k>1 \; and \;  \tau(w_{k-1}^*, w_k^*) \ne 0 \\
        \tau_{k-1}^*, & otherwise
    \end{cases}
\end{equation*}

This ensures that every trend is non-zero. If a zero trend is found with the corrected windows as well, the last non-zero trend is carried forward. The non-zero trends are used to define reversal of trends $r_k$ as $r_1 = 1$, $r_k = sign(\tau_{k-1}^* - \tau_k^*)$ and let

\begin{equation*}
    T^*(k) = \begin{cases}
        \{ \min_t\{l_1, h_1\} \}, & k = 1 \\
        \{ l_{k-1} \}, & r_k = 1 \; and \; k > 1 \\
        \{ h_{k-1} \}, & r_k = -1 \; and \; k > 1 \\
        \varnothing, & r_k = 0 \; and \; k > 1
    \end{cases}
\end{equation*}

This can be used to define a new time series $T^{(1)} = \cup_{k=1}^{n^*} T^*(k) \subset T$ which contain only the points of T where a trend reversal happens.

\subsection{Multiple iterations}

The output of the algorithm, $T^{(1)}$, can then be used for another iteration to get $T^{(2)} \subset T^{(1)}$. This can be repeated until $\#T^{(m)} < s$, at which point we get $T^{(m^*)} = \varnothing \; \forall m^* > m$ and a decreasing sequence of sets $\mathcal{T}$ can be defined as $\mathcal{T} = (T, T^{(1)}, \dots, T^{(m)})$ where $T \supset T^{(1)} \supset, \dots, T^{(m)}$.



\section{State of Structured Streams}

The real problem...

\end{document}
